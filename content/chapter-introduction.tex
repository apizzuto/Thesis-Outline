\chapter{Introduction: High Energy Astrophysics}
\label{sec:intro}
\cleanchapterquote{quote quote.}{Person}{Title}

\section{Cosmic Messengers}
\label{sec:intro:multimessenger}

Humans have a rich history of attempting to reconcile their place in the universe with the behaviors of the cosmos. Predating even the invention of the telescope, humankind has tried to discern meaning from the movement of celestial bodies, and as a result have DEVELOPED a menagerie of SOMETHING: from ptolemaic to Copernican, to bla bla bla.  

We now know that these early attempts at explaining the rules that govern our universe were constructed as one might MAKE SOME ANALOGY ABOUT LOOKING THROUGH BLINDERS OR TRYING TO PAINT A PICTURE OF SOMETHING BY LOOKING AT A SLIVER OF GRASS. Not only is the light visible to naked-eye observers limited to the brightest and nearest sources, but the light that interacts with the optical SOMETHING SOMETHING NERVES SOMETHING is telling but a tiny fraction of the story. The light that we can see without the aid of SOMETHING, or optical light, represents only a small portion of the entire electromagnetic spectrum, and light of different wavelengths reveal different characteristics about the objects which emit them. This is because the energy of photons is related to the physical process in which they were created. 

PARAGRAPH ON LIGHT FROM DIFFERENT WAVELENGTHS FROM MALCOLM LONGAIRS BOOK

PARAGRAPH ABOUT HOW LIGHT DOESNT TELL THE WHOLE STORY, THERE ARE OTHER PARTICLES AND ALSO HOW LIGHT IS AN IMPERFECT MESSENGER

% Just as mapping the sky at different energies of the photoelectric spectrum is
% able to provide important information on the electromagnetic behaviour of observed astronomical objects, remapping the sky using a different messenger altogether could provide a
% totally new perspective in the universe and the interactions taking place within it.
% The work in this thesis concerns the physics of high-energy messengers far above


% These distances are so vast that they are often referred to in the time it
% takes light, the fastest messenger, to travel them. The diameter of our own galaxy, the
% Milky Way, takes over 100,000 years for light to traverse. For thousands of years the only
% messenger known to mankind able to carry information across such large distances was
% light. Studies of light have undergone radical changes along with technological discoveries,
% from observations with our eyes alone to sketch out maps of the sky, to improvements in
% the precision of these maps using lenses in telescopes and creating new detectors sensitive
% to wavelengths outside the visible part of the spectrum or even able to measure single
% photons. 



PICTURE OF NGC 1068 IN VARIOUS WAVELENGTHS, AND THEN ALSO THE TERRA INCOGNITA PLOT



\subsection{Photons and gamma rays}
\label{sec:intro:gammarays}


\subsection{Cosmic rays}
\label{sec:intro:cosmicrays}

\subsection{Neutrinos}
\label{sec:intro:neutrinos}

\section{Sources of High-energy radiation}

